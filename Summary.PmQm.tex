%Pakete;
%A4, Report, 12pt
\documentclass[ngerman,a4paper,12pt]{scrreprt}
\usepackage[a4paper, right=20mm, left=20mm,top=30mm, bottom=30mm, marginparsep=5mm, marginparwidth=5mm, headheight=7mm, headsep=15mm,footskip=15mm]{geometry}

%Papierausrichtungen
\usepackage{pdflscape}
\usepackage{lscape}

%Deutsche Umlaute, Schriftart, Deutsche Bezeichnungen
\usepackage[utf8]{inputenc}
\usepackage[T1]{fontenc}
\usepackage[ngerman]{babel}

%quellcode
\usepackage{listings}

%tabellen
\usepackage{tabularx}

%listen und aufzählungen
\usepackage{paralist}

%farben
\usepackage[svgnames,table,hyperref]{xcolor}

%symbole
\usepackage{latexsym,textcomp}
\usepackage{amssymb}

%font
\usepackage{helvet}
\renewcommand{\familydefault}{\sfdefault}

%durch- und unterstreichen
\usepackage{ulem}

%Abkürzungsverzeichnisse
\usepackage[printonlyused]{acronym}

%Bilder
\usepackage{graphicx} %Bilder
\usepackage{float}	  %"Floating" Objects, Bilder, Tabellen...
\usepackage[space]{grffile} %Leerzechen Problem bei includegraphics
\usepackage{wallpaper} %Seitenhintergrund setzen
\usepackage{transparent} %Transparenz

%Tikz, Mindmaps, Trees
\usepackage{tikz}
\usetikzlibrary{mindmap,trees}
\usepackage{verbatim}

%for
\usepackage{forloop}
\usepackage{ifthen}

%Dokumenteigenschaften
\title{Summary PmQm}
\author{Tobias Blaser}
\date{\today{}, Uster}


%Kopf- /Fusszeile
\usepackage{fancyhdr}
\usepackage{lastpage}

\pagestyle{fancy}
	\fancyhf{} %alle Kopf- und Fußzeilenfelder bereinigen
	\renewcommand{\headrulewidth}{0pt} %obere Trennlinie
	\fancyfoot[L]{\jobname} %Fusszeile links
	\fancyfoot[C]{Seite \thepage/\pageref{LastPage}} %Fusszeile mitte
	\fancyfoot[R]{\today{}} %Fusszeile rechts
	\renewcommand{\footrulewidth}{0.4pt} %untere Trennlinie

%Kopf-/ Fusszeile auf chapter page
\fancypagestyle{plain} {
	\fancyhf{} %alle Kopf- und Fußzeilenfelder bereinigen
	\renewcommand{\headrulewidth}{0pt} %obere Trennlinie
	\fancyfoot[L]{\jobname} %Fusszeile links
	\fancyfoot[C]{Seite \thepage/\pageref{LastPage}} %Fusszeile mitte
	\fancyfoot[R]{\today{}} %Fusszeile rechts
	\renewcommand{\footrulewidth}{0.4pt} %untere Trennlinie
}

\usepackage{changepage}

% Abkürzungen für Kapitel, Titel und Listen
\input{toolsAndCommands/shortcutsListAndChapter}
\input{toolsAndCommands/TextStructuringBoxes}

%links, verlinktes Inhaltsverzeichnis, PDF Inhaltsverzeichnis
\usepackage[bookmarks=true,
bookmarksopen=true,
bookmarksnumbered=true,
breaklinks=true,
colorlinks=true,
linkcolor=black,
anchorcolor=black,
citecolor=black,
filecolor=black,
menucolor=black,
pagecolor=black,
urlcolor=black
]{hyperref} % Paket muss unbedingt als letzes eingebunden werden!

\usepackage{graphicx}
\begin{document}

% Inhaltsverzeichnis
\tableofcontents
\clearpage


\ch{Einführung}
	\expl{Wer sind meine Kunden?}{Mein direkter Kunde und die Kunden meines Kunden (Produktanwender)}
	\expl{Ansatz für optimaleLösungen}{\stdImg{v1.1}{Standardsoftware und Prozesse anpassen oder Individualsoftware entwickeln, die auf meine Prozesse passt.}}
	\stdImg{v1.2}{Kompetenzmatrix}
	\stdImg{v1.3}{Erfolgreiche Innovation}
	\stdImg{v1.4}{Alle drei Aspekte müssen erfüllt werden, können jedoch nie alle gleichzeitig perfekt erfüllt werden.}


\ch{Start- und Abschluss eines Projektes}
	\se{Projektziele}
		\expl{Projekte}{Besitzen Start und Ende. Keine ``Never Ending Stories''!}

		\expl{Warum werden Projekte gestartet?}{
			\uli{
				\li Noch nicht abgedecktes Bedürfnis vorhanden
				\li Umfeld hat sich veränder
				\li Effizienzsteigerung
			}
		}
		%\stdImg{v2.0}{Projektphasen} %scan

		\definition{Guerillia Projekte}{Projekte, die über Nacht aus einer Not heraus entstanden sind und sehr schräg in einer Unternehmung drin stehen. Oftmals mit seltsamer Technologie und zweifelhaften Strukturen aufgebaut. }

		\stdImg{v2.1}{Innovationstreiber}
		\definition{Innovation}{Wenn der Kunde eine Entwicklung nachfragt und wirklich auch haben will.}

		\stdImg{v2.2}{Projekte können gestartet werden, wenn alle W-Fragen beantwortet sind.}
		\stdImg{v2.3}{Beziehungsdreiecke für den erfolgreichen Projektstart}

		\stdImg{v2.5}{Requirementsengineering}

		\definition{Kritikalität}{Beschreibung einer Lösung in seiner Entstehung und seiner Anwendung}
		\stdImg{v2.4}{Kritikalitätsklassen}

		\stdImg{v2.6}{Risiko in einem Projekt}
		Projektbegin: Enthusiasmus gross, Energie auch gross, Risiken auch\\
		Mit Fortschreiten des Projektes: Risiken müssen minimiert werden.
	
		\stdImg{v2.7}{Kosten zum Beheben eines Fehlers}
		\stdImg{v2.8}{Definition von Scopes und Machbarkeit (Feasibility)}
	
		\uli{
			\li Blackfields (Unbekannte Punkte) bei den System Requirements Spezifikationen berücksichtigen und am Anfang des Projektes klären. \ra\ Unbekannte Punkte identifizieren und in die Planung und das Risikomanagement aufnehmen.
			\li Ganz klar definieren, was nicht im Scope des Projektes drin ist.
		}
		
	\se{Managementfähigkeiten}
		\stdImg{v2.9}{Managementfähigkeiten gemäss PMI}
		
	\se{Kostenschätzverfahren}
		\stdImg{v2.10}{}
		
		\uli{
			\li Für Blackfields jeweils mehrere Pläne haben und die Kosten für jede Variante berücksichtigen.
			\li Bei der Planung mit Arbeitspakten darf es keine Abhängigkeiten zwischen den Paketen geben \ra\ nicht planbar.
			\li Immer mehrere Methoden verwenden.
		}
		
	\se{Projektabschluss}
		\stdImg{v2.11}{Projektabschluss}
		\uli{
			\li Projekt ist nicht mit der Validierung fertig. Jedes Projekt hat ein Nachlauf!
			\li Erfahrungen nicht einfach im Kopf lassen, sondern dokumentieren
			\li Leute werden entlassen, Projekt wird abgeschlossen.
		}
		
		\sse{Projekt Abschlussphase}
		\oli{
			\li Produktübergabe
			\li Wartungsvereinbarung
			\li Reintegration der Projektmitarbeiter
			\li Ressourcenauflösung
			\li Nachkalkulation (Erfahrung für Kostenschätzungen in der Zukunft)
			\li Abschlussanalyse
			\li Erfahrungssicherung
			\li Abschlusssitzung
			\li Abschlussbericht
			\li Information
		}
		
		\stdImg{v2.12}{Projektübergabes}



\end{document}
