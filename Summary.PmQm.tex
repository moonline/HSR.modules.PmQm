%Pakete;
%A4, Report, 12pt
\documentclass[ngerman,a4paper,12pt]{scrreprt}
\usepackage[a4paper, right=20mm, left=20mm,top=30mm, bottom=30mm, marginparsep=5mm, marginparwidth=5mm, headheight=7mm, headsep=15mm,footskip=15mm]{geometry}

%Papierausrichtungen
\usepackage{pdflscape}
\usepackage{lscape}

%Deutsche Umlaute, Schriftart, Deutsche Bezeichnungen
\usepackage[utf8]{inputenc}
\usepackage[T1]{fontenc}
\usepackage[ngerman]{babel}

%quellcode
\usepackage{listings}

%tabellen
\usepackage{tabularx}

%listen und aufzählungen
\usepackage{paralist}

%farben
\usepackage[svgnames,table,hyperref]{xcolor}

%symbole
\usepackage{latexsym,textcomp}
\usepackage{amssymb}

%font
\usepackage{helvet}
\renewcommand{\familydefault}{\sfdefault}

%durch- und unterstreichen
\usepackage{ulem}

%Abkürzungsverzeichnisse
\usepackage[printonlyused]{acronym}

%Bilder
\usepackage{graphicx} %Bilder
\usepackage{float}	  %"Floating" Objects, Bilder, Tabellen...
\usepackage[space]{grffile} %Leerzechen Problem bei includegraphics
\usepackage{wallpaper} %Seitenhintergrund setzen
\usepackage{transparent} %Transparenz

%Tikz, Mindmaps, Trees
\usepackage{tikz}
\usetikzlibrary{mindmap,trees}
\usepackage{verbatim}

%for
\usepackage{forloop}
\usepackage{ifthen}

%Dokumenteigenschaften
\title{Summary PmQm}
\author{Tobias Blaser}
\date{\today{}, Uster}


%Kopf- /Fusszeile
\usepackage{fancyhdr}
\usepackage{lastpage}

\pagestyle{fancy}
	\fancyhf{} %alle Kopf- und Fußzeilenfelder bereinigen
	\renewcommand{\headrulewidth}{0pt} %obere Trennlinie
	\fancyfoot[L]{\jobname} %Fusszeile links
	\fancyfoot[C]{Seite \thepage/\pageref{LastPage}} %Fusszeile mitte
	\fancyfoot[R]{\today{}} %Fusszeile rechts
	\renewcommand{\footrulewidth}{0.4pt} %untere Trennlinie

%Kopf-/ Fusszeile auf chapter page
\fancypagestyle{plain} {
	\fancyhf{} %alle Kopf- und Fußzeilenfelder bereinigen
	\renewcommand{\headrulewidth}{0pt} %obere Trennlinie
	\fancyfoot[L]{\jobname} %Fusszeile links
	\fancyfoot[C]{Seite \thepage/\pageref{LastPage}} %Fusszeile mitte
	\fancyfoot[R]{\today{}} %Fusszeile rechts
	\renewcommand{\footrulewidth}{0.4pt} %untere Trennlinie
}

\usepackage{changepage}

% Abkürzungen für Kapitel, Titel und Listen
\input{toolsAndCommands/shortcutsListAndChapter}
\input{toolsAndCommands/TextStructuringBoxes}

%links, verlinktes Inhaltsverzeichnis, PDF Inhaltsverzeichnis
\usepackage[bookmarks=true,
bookmarksopen=true,
bookmarksnumbered=true,
breaklinks=true,
colorlinks=true,
linkcolor=black,
anchorcolor=black,
citecolor=black,
filecolor=black,
menucolor=black,
pagecolor=black,
urlcolor=black
]{hyperref} % Paket muss unbedingt als letzes eingebunden werden!

\usepackage{graphicx}
\begin{document}

% Inhaltsverzeichnis
\tableofcontents
\clearpage


\ch{Einführung}
	\expl{Wer sind meine Kunden?}{Mein direkter Kunde und die Kunden meines Kunden (Produktanwender)}
	\expl{Ansatz für optimaleLösungen}{\stdImg{v1.1}{Standardsoftware und Prozesse anpassen oder Individualsoftware entwickeln, die auf meine Prozesse passt.}}
	\stdImg{v1.2}{Kompetenzmatrix}
	\stdImg{v1.3}{Erfolgreiche Innovation}
	\stdImg{v1.4}{Alle drei Aspekte müssen erfüllt werden, können jedoch nie alle gleichzeitig perfekt erfüllt werden.}


\ch{Start- und Abschluss eines Projektes}
	\se{Projektziele}
		\expl{Projekte}{Besitzen Start und Ende. Keine ``Never Ending Stories''!}

		\expl{Warum werden Projekte gestartet?}{
			\uli{
				\li Noch nicht abgedecktes Bedürfnis vorhanden
				\li Umfeld hat sich veränder
				\li Effizienzsteigerung
			}
		}
		%\stdImg{v2.0}{Projektphasen} %scan

		\definition{Guerillia Projekte}{Projekte, die über Nacht aus einer Not heraus entstanden sind und sehr schräg in einer Unternehmung drin stehen. Oftmals mit seltsamer Technologie und zweifelhaften Strukturen aufgebaut. }

		\stdImg{v2.1}{Innovationstreiber}
		\definition{Innovation}{Wenn der Kunde eine Entwicklung nachfragt und wirklich auch haben will.}

		\stdImg{v2.2}{Projekte können gestartet werden, wenn alle W-Fragen beantwortet sind.}
		\stdImg{v2.3}{Beziehungsdreiecke für den erfolgreichen Projektstart}

		\stdImg{v2.5}{Requirementsengineering}

		\definition{Kritikalität}{Beschreibung einer Lösung in seiner Entstehung und seiner Anwendung}
		\stdImg{v2.4}{Kritikalitätsklassen}

		\stdImg{v2.6}{Risiko in einem Projekt}
		Projektbegin: Enthusiasmus gross, Energie auch gross, Risiken auch\\
		Mit Fortschreiten des Projektes: Risiken müssen minimiert werden.
	
		\stdImg{v2.7}{Kosten zum Beheben eines Fehlers}
		\stdImg{v2.8}{Definition von Scopes und Machbarkeit (Feasibility)}
	
		\uli{
			\li Blackfields (Unbekannte Punkte) bei den System Requirements Spezifikationen berücksichtigen und am Anfang des Projektes klären. \ra\ Unbekannte Punkte identifizieren und in die Planung und das Risikomanagement aufnehmen.
			\li Ganz klar definieren, was nicht im Scope des Projektes drin ist.
		}
		
	\se{Managementfähigkeiten}
		\stdImg{v2.9}{Managementfähigkeiten gemäss PMI}
		
	\se{Kostenschätzverfahren}
		\stdImg{v2.10}{}
		
		\uli{
			\li Für Blackfields jeweils mehrere Pläne haben und die Kosten für jede Variante berücksichtigen.
			\li Bei der Planung mit Arbeitspakten darf es keine Abhängigkeiten zwischen den Paketen geben \ra\ nicht planbar.
			\li Immer mehrere Methoden verwenden.
		}
		
	\se{Projektabschluss}
		\stdImg{v2.11}{Projektabschluss}
		\uli{
			\li Projekt ist nicht mit der Validierung fertig. Jedes Projekt hat ein Nachlauf!
			\li Erfahrungen nicht einfach im Kopf lassen, sondern dokumentieren
			\li Leute werden entlassen, Projekt wird abgeschlossen.
		}
		
		\sse{Projekt Abschlussphase}
		\oli{
			\li Produktübergabe
			\li Wartungsvereinbarung
			\li Reintegration der Projektmitarbeiter
			\li Ressourcenauflösung
			\li Nachkalkulation (Erfahrung für Kostenschätzungen in der Zukunft)
			\li Abschlussanalyse
			\li Erfahrungssicherung
			\li Abschlusssitzung
			\li Abschlussbericht
			\li Information
		}
		
		\stdImg{v2.12}{Projektübergabes}


\ch{Risikomanagement}
	Jedes Projekt birgt Risiken (Aufgaben, Zustände, Umstände die nicht schlüssig bewertet werden können).
	
	\expl{Rückstellungen zum Abfedern von Risiken in Unternehmen}{Geld ung Zeit}

	Risiken sind Unsicherheitsfaktoren, die schlussendlich immer einen Schaden zur Folge haben!
	
	\important{Dieser Schaden muss in Kauf genommen werden!}
	\examp{Schaden, der in Kauf genommen wird}{Blechschaden beim Autofahren. Risikomanagement: Knautschzonen, Airbag, Überrollschutz, ...}
	
	\stdImg{v3.1}{Arten von Risiken}
	
	\se{Projektrisikomanagement}
		\stdImg{v3.2}{Eintrittswahrscheinlichkeit und Schadensausmass von Risiken}
	
		Alles was rot ist, sollte man 100\% kontrollieren können!
	
		\stdImg{v3.3}{Massnahmen zum pro-aktiven Risiko Management}
		
		\uli{
			\li Vermindern: Automatisierung ist Risikomanagement (Risiko Mensch ausschliessen)
			\li Vermindern: Was man tut richtig tun!
			\li ART: Risiko weitergeben / rückversichern
			\li Risiko selbst tragen: Rückstellung auf meinem Bankkonto
		}
		
		\stdImg{v3.4}{Risiko Management Prozess}		
		Projektrisikomanagement ist ein kontinuierlicher Prozess
		
		\sse{Elemente des Risikomanagements}
		
			Risiko Einschätzung
			\uli{
				\li Risiko Identifikation
				\li Risiko Analyse
				\li Risiko Prioritätenbildung
			}
				
				
			Risiko Kontrolle
			\uli{
				\li Risiko Management Planung (Risiken können sich über der Zeit verändern)
				\li Risiko Lösung
				\li Risiko Überwachung
			}
			
			
		\sse{Risikoidentifikation}
			\stdImg{v3.5}{Risiken identifizieren}
			
			\stdImg{v3.6}{Risiko Bewertung}
			
			\uli{
				\li Beispiel Risikoverminderung Projektdauer: Bereits nach wenigen Monaten eine Release mit eingeschränkten Features herausbringen.
				\li Nutzen für das Unternehmen: Immer eine Nutzenbetrachtung machen
				\li Geschäfftskenntnisse: Zumindest der Projektleiter sollte sie haben
				\li Anforderungen: So einfach wie möglich aber nicht einfacher
				\li Nie mit einem Projekt beginnen, wenn unbekannt ist wer das Projekt bezahlt
			}

			\stdImg{v3.7}{Risiko Bewertung}
			
			\uli{
				\li Kunde muss unbedingt Zeit haben, um das Projekt zu begleiten
				\li Viele Änderungen: Anforderungen wurden nicht genug sauber definiert
				\li Änderungen in der Organisationsstruktur des Kunden: So schnell wie möglich den neuen Verantwortlichen kennenlernen oder wenn häaufige Änderungen: Projekt sistieren bis sich die Lage beruhigt hat.
			}
			
		\sse{Risikoanalyse}
			\uli{
				\li Abschätzung der Kostenkonsequenz des Risikos.
				\li Im Projektbudget gibt es eine Position Risiko (Rückstellungen), die von niemandem angetastet werden darf.
				\li Benale (Strafen) müssen ins Budget aufgenommen werden!
			}
			\stdImg{v3.8}{Risiko Kosten Einschätzung}
			
		\sse{Risikopriorisierung}
			\uli{
				\li Konzentration auf die wichtigsten Risiken
				\li Risiken nicht als Ausreden benutzen!
			}

		\sse{Risiko Lösungsstrategien}
			\stdImg{v3.9}{}
			
			
\ch{Wirtschaftlichkeitsrechnung}
	\se{Projektkosten}
		\stdImg{v4.1}{Elemente der Projektkosten}
		
	\se{Projektnutzen}
		\uli{
			\li Primärnutzen: direktes neues Feature, neue Einnahmen
			\li Primärnutzen: Ich kann Ausgaben einsparen
		}
		\stdImg{v4.2}{Elemente des Projektnutzen}
		\uli{
			\li Substituierung: Ersetzung einer Firma durch andere \ra\ Beispiel Kodak wurde substituiert durch Andere Firmen/Produkte weil Kameras plötzlich keine Filme mehr benötigen
			\li Einmahllizenzen ermöglichen jährliche Wartungsvertäge, die regelmässig Geld in die Kasse spülen
		}
		
	\se{Kosten-Nutzen}
		\uli{
			\li Der Nutzen/Einsparung bei einem Projekt sollte viel mehr als die Kosten sein, da die Projekte immer teurer werden.
			\li Effektive Produktionskosten oft erst nach dem Projekt bekannte \ra\ Projekt muss wesentlich mehr Nutzen bringen als Kosten
			\li Wenn die Betriebskosten und die Projektkosten start steigen (hoch sind), liegt viel Druck auf dem Projekt und wenn das Projekt länger läuft als geplant wird viel mehr Geld verbraucht.
		}
		
	\se{Liquidität}
		\uli {
			\li Genügend Geld jederzeit verfügbar haben
			\li Liquidität ist wie Atemluft (fehlt sie, ist das Projekt am Ende), Gewinn ist wie Essen (man kommt auch eine Zeit lang ohne aus)
			\li Zahlungsmodalitäten
				\uli{
					\li Festpreis mit gestaffelter Zahlung
						\stdImg{v4.3}{Festpreis}
					\li Zahlungsplan nach Aufwand (z.B. monatlich)
						\stdImg{v4.4}{Zahlung nach Aufwand}
						Wird zu spät abgerechnet, fehlt konstant Geld! \ra\ Braucht vor Allem am Anfang eine eigene Vorfinanzierung!
					\li Zahlung nach Projekterfolg (z.B. Meilensteine)
						\stdImg{v4.5}{nach Milestones}
						Braucht ebenfall eine Vorfinanzierung.
				}
			\li Bezahlt der Kunde zu spät (60 Tage statt 30 Tage), so brauche ich die doppelte Vorfinanzierung! \ra\ Bei der Liquiditätsplanung Valuta und nicht Rechnungsdatum als Zeitpunkt einplanen.
			\li Kunden, die nicht rechtzeitig bezahlen sind oft unzufrieden oder in Schwierigkeiten.
		}
		
		
\ch{Project Controlling}
	\uli{
		\li Kein Projekt besteht nur aus dem Softwareentwicklungsprozess
		\li Andere Komponenten dürfen nicht vergessen werden
	}
	\stdImg{v5.1}{Project Controller}
	\uli{
		\li Controller hilft zu Entscheiden und Entscheide mitzutragen
		\li Controller reflektiert Projekt
	}
	\expl{Controlling}{überwacht den Status des Projektes}
	
	\stdImg{v5.2}{Projektstatusüberwachung}
	\expl{Steering Commitee}{Ausserprojektinstanz, die ein Nutzen vom Projekt haben oder zusätzliche Informationen einbringen. (Wie in einem Betrieb der Verwaltungsrat) Übernimmt in einem Projekt die übergeordnete Verantwortung.}
	\stdImg{v5.3}{Controlling Aufgaben}
	\uli{
		\li Controlling verifiziert die vom Projektleiter gesetzten Status Flaggs und informiert das Steering Commitee.
		\li Sponsor ist auch Mitglied des Steering Commitees.
	}
	
	\se{Standards}
		\stdImg{v5.4}{Standards sollen Vorteile bringen}
		
	\se{Cockpit (Ist / Soll)}
		\uli{
			\li Forcast: Anhand des aktuellen Projektstands den Terminplan anpassen und die Arbeiten anpassen
			\li Ist ein Meilenstein verspätet weil zusätzliche Changes reinkommen, so wird der verspätete Termin zum Plan \ra\ Replanning
		}
		\stdImg{v5.5}{Ampelprinzip}
		

\ch{SCRUM}
			
	\stdImg{v6.0}{Wasserfall vs Iterativer Ansatz}
	\stdImg{v6.1}{Dealing with Uncertainty and Risks : Hundekurs}
	\stdImg{v6.2}{Was ist Scrum}	
	
	\stdImg{v6.3}{Timeboxing, Sprints, Inspect \& Adapt}
	Das Team schätzt den Aufwand für eine Anforderung, nicht der Auftraggeber!
	
	\expl{Daily Scrum Meeting}{Kein Palaver. Jeder hat 2 Minuten für drei Dinge: Was habe ich in den letzten 24h gemacht, was mache ich heute und was hindert mich / brauche ich noch?}
	
	\stdImg{v6.4}{Scrum Elements}
	\uli{
		\li	Product Owner ist kein Projektleiter! Owner weis was er will, was gemacht werden soll. Vertritt den Auftraggeber. Sollte nicht Teil des Teams sein. Idealerweise jemand vom Kunden. Muss in hohem Mass verfügbar sein.
		\li Team: Zuständig für die Umsetzung.
		\li Scrum Master: Kommt meisst aeisst aus dem Team oder von extern. Ist kein Chef! Ist nur ein Mediator. Schaut, dass der Scrum Prozess läuft. Schaut, das die Artefakte von denjenigen gemacht werden, die dafür zuständig sind.
		\li Ist ein per se anti-hierarchisches Modell!
		\li Alles ist vollständig transparent und jeder kann Lösungen zu allem einbringen.
	}
	
\ch{ITIL}
	\stdImg{v7.2}{}
	\uli{
		\li Projektziele: 
		\uli{
			\li Kunde zufrieden
			\li Budget eingehalten
		}
		\li Ziele von IT Service Management: 
			\uli{
				\li zuverlässigen IT Service mit hoher Qualität und tragbaren Kosten
				\li Trennung von Entwicklung und Betrieb
				\li \ra\ Verhindern von vorschnellen Änderungen, die den Betrieb stören
			}
	}

	\se{Erfolgsfaktoren}
	\stdImg{v7.1}{Erfolgsfaktoren}
		\uli{
			\li Es muss klar sein was der Kunde erwarten kann und was wir überhaupt liefern
			
		}
		\stdImg{v7.3}{}
		
	\se{Business Nutzen}
		\stdImg{v7.4}{}
		System nütz nur etwas, wenn es auch vernünftig läuft. Die tollste Funktion ist nutzlos wenn das System nicht läuft.
		
	\se{Qualität - Nutzen - Design}
		\stdImg{v7.5}{}
		
	\se{Service Management}
		\stdImg{v7.6}{}
		
	\se{Capacity Management}
		\stdImg{v7.7}{}
		\uli{
			\li Latenz
			\li Bandbreite
		}
		
	\se{Availability Management}
		System Design, das es die geforderte Verfügbarkeit erreicht.
		\stdImg{v7.8}{}
		
	\se{IT Service Continuity Management}
		\stdImg{v7.9}{}
		Schutz gegen Aussen (gelb) muss immer noch gewährleistet sein!
		
	\se{Service Transition}
		Überführung von dem was wir entwickelt haben in die Operation.
		\stdImg{v7.10}{}
		Kunde hat erst einen Nutzen wenn der Service läuft! Vom Projektmanagemet hat er nichst!		
		
	\se{Change Management}
		\stdImg{v7.11}{}
		\stdImg{v7.12}{Release \& Deploy Management}
		\uli{
			\li DML Dient dazu sicherzustellen, das alle vom gleichen Ausgehen. Wenn direkt vom Pilot zum Rollout übernommen wird, kann es sein das noch was verändert wurde.
			\li Wichtige Meilensteine sollten nicht vom Team freigegeben werden sondern vom Operator
		}
		\stdImg{v7.13}{Was gehört zum Release und Deploymanagement?}
		Wenn die Operation die Operationdokumentation selbst schreibt, versteht sie sie sicher.

	\se{Service Validation \& Testing}
		\stdImg{v7.14}{}

	\se{Service Asset und Configuration Management}
		\stdImg{v7.15}{}
		
	\se{Monitoring}
		Jemand muss die Events auch erhalten und darauf reagieren können!
		\stdImg{v7.16}{}
		
		
	\se{Problem Management}
		\definition{Problem}{Störung unbekannten ausmasses.}
		\stdImg{v7.17}{}
		
		
\ch{Prozess Modelle}
	\expl{Zweck Prozessmodelle}{Prozessmodelle braucht es um grundlegende Fragen abzuklären.
		\uli{
			\li Wie lange man brauchen wird
			\li Wie viele Leute notwendig sind
			\li Was das Resultat sein wird
			\li In welchen Schritten vorgegagen wird und welche Teilresultate es gibt
		}
	}
		
	\stdImg{v8.1}{}
	
	\se{Wasserfallmodell}
		\stdImg{v8.2}{Wasserfallmodell}
		\uli{
			\li vom Kleinen ins Grosse
			\li schrittweise Innovation
			\li Gut für anfassbare Produkte
			\li Jeder Schritt wird in der vollen Breite durchgeführt
			\li 
		}
		\stdImg{v8.3}{Wasserfallmodell Ablauf}

	\se{V-Modell}
		\stdImg{v8.4}{V-Modell}
		\uli{
			\li ähnlich wie das Wasserfallmodell
			\li links wird gearbeitet, rechts getestet
			\li Bei der Festlegung der Anforderung wird auch gleich die Abnahme festgelegt
			\li Unterschied zum Wasserfall modell: Spezifikation der Abnahmetests und Planung der Strategie
			\li Testing geht vom KLeinen ins Grosse
		}
		\expl{Unterschied Verifikation und Validation}{Verifikation: überprüfung ob das was man tut das Richtige ist, Validation: Abnahme stellt auch rechtlicher Abschluss des Projektes dar}
		\stdImg{v8.5}{V-Modell normierte Darstellung}

	\se{Prototypen Modell}
		Ziel: Risken senken, Dem Kunden zeigen das es funktioniert, zeigen, das Parameter eingehalten werden, Kunden für ein neues Projekt gewinnen, ... .
		\stdImg{v8.6}{Prototypen Modell}

	\se{Evolutionäres Modell}
		\stdImg{v8.7}{Evolutionäres Modell}
		\uli{
			\li Stufenweise / Schrittweise Entwicklung
			\li Anforderungen entwickeln sich mit
			\li Wartung ist teil der Weiterentwicklung
			\li Eignet sich gut, wenn der Auftraggeber noch nicht alle Anforderungen kennt
			\li Code driven Development
		}

	\se{Inkrementelles Modell}
		Fast gleich wie evolutionäres. Unterschied ist nur, das fertige (zwischen)Releases entwickelt werden, die auch ausgeliefert werden könnten.
		\stdImg{v8.8}{Inkrementelles Modell}
		
	\se{Nebenläufiges Modell}
		In schneller Zeit Projekt durchführen weil z.B. Elektro, Software und Pneumatikteil paralle entwickelt werden kann.
		\stdImg{v8.9}{}
		Braucht immer ein Lead (Hauptprojekt)
		
	\se{Objektorientiertes Modell}
		\stdImg{v8.10}{Objektorientiertes Modell}
		\uli{
			\li Kommen Interfaces in grösseren Anwendungen vor, macht das Modell sehr Sinn
			\li Entwicklung wiederverwendbarer Module
			\li Braucht guten Verwaltungsapparat, der die Module managed (Releasemanagement)
			\li Standards im Projekt sehr wichtig!
		}
		
	\se{Spiralen Modell}
		\uli{
			\li Idee fängt klein und einfach an
			\li Mit der fortschreitenden Entwicklung wächst der Einsatzbereich und der Umfang des Produktes
			\li Releases gewähren stabile Versionen
		}
		\stdImg{v8.11}{Spiralmodell}


\ch{Configuration Management}
	Verhindert Probleme beim Deployment durch vergessene Konfiguration
	
	\stdImg{v9.1}{Configuration Management muss von Begin weg starten}
	
	\expl{Quellenelement und abgeleitetes Element}{Quelle: Sourcecode, Abgeleitetes: Compilat\\
		Will der Kunde unbedingt den Source Code, so ist es unabdingbar vertraglich sich von der Verwantwortung das die Software läuft zu entbinden und dem Kunden Vertraglich Änderungen an der Software zu verbieten.
	}
	
	\uli{
		\li Version: Schritt in der Entwicklung (Intern)
		\li Release: Ausgerollte abgegebene fertige Version
	}

	V1.2.3:
	\uli{
		\li 1: Release
		\li 2: Version
		\li 3: Build/Update
	}
	\stdImg{v9.3}{Releases, Versionen und Fixes}
	
	\stdImg{v9.4}{Softwarevarianten}
	Varianten sind sehr gefährlich!!
	
	
\ch{Erfolgsfaktoren um ein Projekt erfolgreich zum Abschluss zu bringen}
	\uli{
		\li Lompetenz der Mitarbeiter, Wer kann was, wer ist wie aufgestellt
		\li Scope / BRS
		\li Projektgrösse
		\li Planung
		\li Kundenzufriedenheit (Verhältnis)
		\li Teamzusammenarbeit
	}
	
	
\ch{Grundlagen des Managements}
	\stdImg{v10.1}{Von der Vision zur Aktion}
	\uli{
		\li Strategie: Rahmenbedingungen. Z.b. festlegen wann gestartet wird.
	}
	\stdImg{v10.2}{}
	\uli{
		\li Massnahmen einleiten, um neue Bedürfnisse rechtzeitig erkennen zu können und in das Projekt einfliessen lassen können
	}
	
	\stdImg{v10.3}{}
	Wer nicht so erfahren ist und nicht so selbstständig ist, erzeugt einen grösseren Managementaufwand.

	\stdImg{v10.4}{}
	\uli{
		\li Umfeld hat starken Einfluss auf Projekt
		\li Interessen des Umfeldes können zu Marketing Zwecken verwendet werden
	}
	\stdImg{v10.5}{Stakeholders}

	\se{Kommunikation}
		\uli{
			\li Mitarbeiter sollten nie direkt nach aussen kommunizieren
			\li Kommunikation klar definieren
			\li Am Schlimmsten ist es, wenn MA zu falschen Stellen nach Aussen unterschiedliche Informationen kommunizieren.
		}
		

\ch{Normen}
	\se{Warum Normen?}
	\uli{
		\li Kosten sparen
		\li Effizient arbeiten
		\li Monopolen vorbeugen
		\li Austauschbarkeit
		\li Zusammenarbeit
		\li Gesetzliche Normenvorgabe
	}
	\expl{Normen}{planmässige gemeinheitliche Vereinheitlichung von materiellen Gegenständen}

	\se{Nutzen vom Erfüllen einer Norm bsp. ISO 27000? oder DoD 498}
	\uli{
		\li Ausschreiber schreibt dies vor (Wettbewerbsvorteil)
		\li Prämiensenkung Versicherung
		\li Auftraggeber will mich auditieren bevor ich seinen Auftrag abwickle. Durch das Erfüllen meiner Norm kann ich ihn mit dem genormten letzten Audit abspeisen und muss nicht meine Firma und mein Wissen offenlegen.
	}
	








\end{document}
